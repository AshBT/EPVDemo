\documentclass{article}\usepackage[]{graphicx}\usepackage[]{color}
%% maxwidth is the original width if it is less than linewidth
%% otherwise use linewidth (to make sure the graphics do not exceed the margin)
\makeatletter
\def\maxwidth{ %
  \ifdim\Gin@nat@width>\linewidth
    \linewidth
  \else
    \Gin@nat@width
  \fi
}
\makeatother

\definecolor{fgcolor}{rgb}{0.345, 0.345, 0.345}
\newcommand{\hlnum}[1]{\textcolor[rgb]{0.686,0.059,0.569}{#1}}%
\newcommand{\hlstr}[1]{\textcolor[rgb]{0.192,0.494,0.8}{#1}}%
\newcommand{\hlcom}[1]{\textcolor[rgb]{0.678,0.584,0.686}{\textit{#1}}}%
\newcommand{\hlopt}[1]{\textcolor[rgb]{0,0,0}{#1}}%
\newcommand{\hlstd}[1]{\textcolor[rgb]{0.345,0.345,0.345}{#1}}%
\newcommand{\hlkwa}[1]{\textcolor[rgb]{0.161,0.373,0.58}{\textbf{#1}}}%
\newcommand{\hlkwb}[1]{\textcolor[rgb]{0.69,0.353,0.396}{#1}}%
\newcommand{\hlkwc}[1]{\textcolor[rgb]{0.333,0.667,0.333}{#1}}%
\newcommand{\hlkwd}[1]{\textcolor[rgb]{0.737,0.353,0.396}{\textbf{#1}}}%

\usepackage{framed}
\makeatletter
\newenvironment{kframe}{%
 \def\at@end@of@kframe{}%
 \ifinner\ifhmode%
  \def\at@end@of@kframe{\end{minipage}}%
  \begin{minipage}{\columnwidth}%
 \fi\fi%
 \def\FrameCommand##1{\hskip\@totalleftmargin \hskip-\fboxsep
 \colorbox{shadecolor}{##1}\hskip-\fboxsep
     % There is no \\@totalrightmargin, so:
     \hskip-\linewidth \hskip-\@totalleftmargin \hskip\columnwidth}%
 \MakeFramed {\advance\hsize-\width
   \@totalleftmargin\z@ \linewidth\hsize
   \@setminipage}}%
 {\par\unskip\endMakeFramed%
 \at@end@of@kframe}
\makeatother

\definecolor{shadecolor}{rgb}{.97, .97, .97}
\definecolor{messagecolor}{rgb}{0, 0, 0}
\definecolor{warningcolor}{rgb}{1, 0, 1}
\definecolor{errorcolor}{rgb}{1, 0, 0}
\newenvironment{knitrout}{}{} % an empty environment to be redefined in TeX

\usepackage{alltt}
\usepackage{fullpage}
\usepackage{booktabs}
\usepackage{amsthm,amsmath,amssymb}
\RequirePackage{natbib}
\usepackage{graphicx}

\title{EPV Demo \\
\Large
Supplement to ``A Multiresolution Stochastic Process Model for Predicting Basketball Possession Outcomes''}
\author{Daniel Cervone, Alex D'Amour, Luke Bornn and Kirk Goldsberry}
\date{}
\IfFileExists{upquote.sty}{\usepackage{upquote}}{}
\begin{document}

\maketitle

This document provides a demonstration of the code, methodology, and inferential results for the EPV model discussed in the paper. 

\section{Loading the Data}

To begin, we must first set the directories containing the supplemental data and code.

\begin{knitrout}
\definecolor{shadecolor}{rgb}{0.969, 0.969, 0.969}\color{fgcolor}\begin{kframe}
\begin{alltt}
\hlstd{code.dir} \hlkwb{<-} \hlstr{"~/xyhoops/XYHoops/dlc_src/new/demo/code"}
\hlstd{data.dir} \hlkwb{<-} \hlstr{"~/xyhoops/XYHoops/dlc_src/new/demo/data"}
\end{alltt}
\end{kframe}
\end{knitrout}

Now we load the \texttt{csv} file containing a full game of optical tracking data:

\begin{knitrout}
\definecolor{shadecolor}{rgb}{0.969, 0.969, 0.969}\color{fgcolor}\begin{kframe}
\begin{alltt}
\hlstd{dat} \hlkwb{<-} \hlkwd{read.csv}\hlstd{(}\hlkwc{file}\hlstd{=}\hlkwd{sprintf}\hlstd{(}\hlstr{"%s/2013_11_01_MIA_BKN.csv"}\hlstd{, data.dir))}
\end{alltt}
\end{kframe}
\end{knitrout}

Each row of \texttt{dat} represents a time point (sampled 25 times per second), and columns include
\begin{table}[!h]
\begin{center}
\begin{tabular}{r|ll}
\toprule
Column & Value & Notes \\
\midrule
\texttt{time} & Real time (ms) & \\
\texttt{game} & Game ID & \\
\texttt{quarter} & Quarter & \\
\texttt{shot\_clock} & Time remaining on shot clock & \\
\texttt{game\_clock} & Time remaining in quarter (s) & \\
\texttt{x, y, z} & Ball position (ft) & Court region is $[0, 94] \times [0, 50]$ \\
\texttt{a1\_ent} & ID number of player 1 on away team (\texttt{a1}) & \\
\texttt{a1\_x, a1\_y} & Position of \texttt{a1} & \\
\texttt{a1\_event} & Event code for player \texttt{a1} &  See Table \ref{tab:event_codes} for reference \\
\texttt{a\#\_*, h\#\_*} & As for \texttt{a1} & \\
\bottomrule
\end{tabular}
\caption{Description of variables in optical tracking data sample.}
\label{tab:data_desc}
\end{center}
\end{table}

Let's plot the data for some arbitrary moment in the game in Figure \ref{fig:plot_data}.

\begin{knitrout}
\definecolor{shadecolor}{rgb}{0.969, 0.969, 0.969}\color{fgcolor}\begin{kframe}
\begin{alltt}
\hlkwd{source}\hlstd{(}\hlkwd{sprintf}\hlstd{(}\hlstr{"%s/graphics.R"}\hlstd{, code.dir))}
\hlkwd{par}\hlstd{(}\hlkwc{mar}\hlstd{=}\hlkwd{c}\hlstd{(}\hlnum{0}\hlstd{,}\hlnum{0}\hlstd{,}\hlnum{0}\hlstd{,}\hlnum{0}\hlstd{))}
\hlkwd{data.plotter}\hlstd{(dat,} \hlnum{1800}\hlstd{)}
\end{alltt}
\end{kframe}\begin{figure}[h!]

{\centering \includegraphics[width=\maxwidth]{figure/plot_data-1} 

}

\caption[Plotting a single moment of optical tracking data]{Plotting a single moment of optical tracking data.}\label{fig:plot_data}
\end{figure}


\end{knitrout}

In this format, the data lacks information necessary for computing EPV. Most importantly, the identity of the ballcarrier is not labeled, and most be inferred by the record of game actions (and positional data). We also need to record the covariates used by our multiresolution transition models, and perform some simple data manipulations, such as rotating all data to the offensive half-court. The following code performs these data tasks:

\begin{knitrout}
\definecolor{shadecolor}{rgb}{0.969, 0.969, 0.969}\color{fgcolor}\begin{kframe}
\begin{alltt}
\hlkwd{source}\hlstd{(}\hlkwd{sprintf}\hlstd{(}\hlstr{"%s/data_formatting.R"}\hlstd{, code.dir))}
\hlkwd{source}\hlstd{(}\hlkwd{sprintf}\hlstd{(}\hlstr{"%s/covariates.R"}\hlstd{, code.dir))}

\hlstd{poss} \hlkwb{<-} \hlkwd{possession.indicator}\hlstd{(dat)} \hlcom{# infer ballcarrier... takes about a minute}
\hlstd{new.dat} \hlkwb{<-} \hlkwd{rearrange.data}\hlstd{(dat, poss)} \hlcom{# re-shuffle columns by to ballcarrier... (2 min)}
\hlstd{new.dat} \hlkwb{<-} \hlkwd{offensive.halfcourt}\hlstd{(new.dat)} \hlcom{# transforming to offensive halfcourt}
\hlstd{covariates} \hlkwb{<-} \hlkwd{getAllCovars}\hlstd{(new.dat)} \hlcom{# get covariates... (3 min)}
\hlstd{new.dat} \hlkwb{<-} \hlkwd{data.frame}\hlstd{(new.dat, covariates)}
\hlkwd{save}\hlstd{(new.dat,} \hlkwc{file}\hlstd{=}\hlkwd{sprintf}\hlstd{(}\hlstr{"%s/new.dat.Rdata"}\hlstd{, data.dir))}
\end{alltt}
\end{kframe}
\end{knitrout}

Or, since this takes few minutes to complete, it may be easier to load a pre-computed version of the transformed data set, \texttt{new.dat}:
\begin{knitrout}
\definecolor{shadecolor}{rgb}{0.969, 0.969, 0.969}\color{fgcolor}\begin{kframe}
\begin{alltt}
\hlkwd{load}\hlstd{(}\hlkwd{sprintf}\hlstd{(}\hlstr{"%s/new.dat.Rdata"}\hlstd{, data.dir))}
\end{alltt}
\end{kframe}
\end{knitrout}

\section{Components of hierarchical models}

The hierarchical models used to estimate parameters for the multiresolution transition models rely on preprocessed data summaries. First, the conditional autoregressive model priors used for many model parameters rely on a graph $\mathbf{H}$ of player similarity. As discussed in the paper, this graph is constructed based on the similarity in players' court occupancy distributions. We can visualize these court occupancy distributions, as well as the similarity scores we calculate between them.

\begin{knitrout}
\definecolor{shadecolor}{rgb}{0.969, 0.969, 0.969}\color{fgcolor}\begin{kframe}
\begin{alltt}
\hlkwd{load}\hlstd{(}\hlkwd{sprintf}\hlstd{(}\hlstr{"%s/playerbases.Rdata"}\hlstd{, data.dir))}
\hlstd{players} \hlkwb{<-} \hlkwd{read.csv}\hlstd{(}\hlkwd{sprintf}\hlstd{(}\hlstr{"%s/players2013.csv"}\hlstd{, data.dir))}
\hlkwd{head}\hlstd{(players)}
\end{alltt}
\begin{verbatim}
##   X player_id firstname lastname       position height weight byear rookie
## 1 1      3306     Elton    Brand Forward-Center     81    254  1979   1999
## 2 2     58293      Kyle   Korver  Guard-Forward     79    212  1981   2003
## 3 3    292401       Lou Williams          Guard     73    175  1986   2005
## 4 4    237675      Paul  Millsap Forward-Center     80    258  1985   2006
## 5 5    280587        Al  Horford Forward-Center     82    250  1986   2007
## 6 6    398043      Jeff   Teague    Point-Guard     74    181  1988   2009
\end{verbatim}
\end{kframe}
\end{knitrout}

\texttt{players} is a directory of the 461 NBA players in the 2013-14 season, and \texttt{playerbases.Rdata} contains summaries of their court occupancy patterns. \texttt{df} is the matrix $\mathbf{G}$ from the paper: plotting its rows reveals stark differences in players' spatial occupancy patterns:

\begin{knitrout}
\definecolor{shadecolor}{rgb}{0.969, 0.969, 0.969}\color{fgcolor}\begin{kframe}
\begin{alltt}
\hlkwd{par}\hlstd{(}\hlkwc{mfrow}\hlstd{=}\hlkwd{c}\hlstd{(}\hlnum{1}\hlstd{,}\hlnum{5}\hlstd{))}
\hlkwa{for}\hlstd{(i} \hlkwa{in} \hlnum{1}\hlopt{:}\hlnum{5}\hlstd{)}
  \hlkwd{spatialPlot0}\hlstd{(df[i, ],} \hlkwc{legend}\hlstd{=F)}
\end{alltt}
\end{kframe}\begin{figure}[h!]

{\centering \includegraphics[width=\maxwidth]{figure/plot_occupancy-1} 

}

\caption[Court occupancy distributions]{Court occupancy distributions.}\label{fig:plot_occupancy}
\end{figure}


\end{knitrout}

In the paper, we use non-negative matrix factorization to obtain a rank 5 approximation of the court occupancy distribution matrix. The basis surfaces of this approximation, given in Figure 8 of the paper, are reproduced here:

\begin{knitrout}
\definecolor{shadecolor}{rgb}{0.969, 0.969, 0.969}\color{fgcolor}\begin{kframe}
\begin{alltt}
\hlkwd{par}\hlstd{(}\hlkwc{mfrow}\hlstd{=}\hlkwd{c}\hlstd{(}\hlnum{1}\hlstd{,}\hlnum{5}\hlstd{))}
\hlkwa{for}\hlstd{(i} \hlkwa{in} \hlnum{1}\hlopt{:}\hlnum{5}\hlstd{)}
  \hlkwd{spatialPlot0}\hlstd{(nmf.basis[i, ],} \hlkwc{legend}\hlstd{=F)}
\end{alltt}
\end{kframe}\begin{figure}[h!]

{\centering \includegraphics[width=\maxwidth]{figure/plot_occupancy_bases-1} 

}

\caption[Court occupancy distribution bases]{Court occupancy distribution bases.}\label{fig:plot_occupancy_bases}
\end{figure}


\end{knitrout}

Projected onto this basis, the court occupancy distributions shown in Figure \ref{fig:plot_occupancy} look like:
\begin{knitrout}
\definecolor{shadecolor}{rgb}{0.969, 0.969, 0.969}\color{fgcolor}\begin{kframe}
\begin{alltt}
\hlstd{df.lowrank} \hlkwb{<-} \hlstd{nmf.coef} \hlopt \hlstd{nmf.basis}
\hlkwd{par}\hlstd{(}\hlkwc{mfrow}\hlstd{=}\hlkwd{c}\hlstd{(}\hlnum{1}\hlstd{,}\hlnum{5}\hlstd{))}
\hlkwa{for}\hlstd{(i} \hlkwa{in} \hlnum{1}\hlopt{:}\hlnum{5}\hlstd{)}
  \hlkwd{spatialPlot0}\hlstd{(df.lowrank[i, ],} \hlkwc{legend}\hlstd{=F)}
\end{alltt}
\end{kframe}\begin{figure}[h!]

{\centering \includegraphics[width=\maxwidth]{figure/appx_occupancy-1} 

}

\caption{Low rank court occupancy distributions for players shown in Figure \ref{fig:plot_occupancy}.}\label{fig:appx_occupancy}
\end{figure}


\end{knitrout}

It's better to compute player similarity using distance in the space of basis loadings, rather than the original court occupancy distributions, as such distances are calculated across axes that best describe player variation. We calculate \texttt{K}, a distance matrix comparing the loadings for the court occupancy distributions of all 461 players, then map this to a symmetric adjacency matrix \texttt{H} based on finding each player's closest eight neighbors:

\begin{knitrout}
\definecolor{shadecolor}{rgb}{0.969, 0.969, 0.969}\color{fgcolor}\begin{kframe}
\begin{alltt}
\hlstd{K} \hlkwb{<-} \hlkwd{matrix}\hlstd{(}\hlnum{NA}\hlstd{,} \hlkwc{nrow}\hlstd{=}\hlkwd{nrow}\hlstd{(df),} \hlkwc{ncol}\hlstd{=}\hlkwd{nrow}\hlstd{(df))}
\hlkwa{for}\hlstd{(i} \hlkwa{in} \hlnum{1}\hlopt{:}\hlkwd{nrow}\hlstd{(K))\{}
  \hlstd{this.coef} \hlkwb{<-} \hlstd{nmf.coef[i, ]} \hlopt{/} \hlkwd{sum}\hlstd{(nmf.coef[i, ])}
  \hlstd{K[i, ]} \hlkwb{<-} \hlkwd{apply}\hlstd{(nmf.coef,} \hlnum{1}\hlstd{,} \hlkwa{function}\hlstd{(}\hlkwc{r}\hlstd{)} \hlkwd{sum}\hlstd{((r} \hlopt{/} \hlkwd{sum}\hlstd{(r)} \hlopt{-} \hlstd{this.coef)}\hlopt{^}\hlnum{2}\hlstd{))}
\hlstd{\}}

\hlstd{H} \hlkwb{<-} \hlnum{0} \hlopt{*} \hlstd{K}
\hlkwa{for}\hlstd{(i} \hlkwa{in} \hlnum{1}\hlopt{:}\hlkwd{nrow}\hlstd{(H))\{}
  \hlstd{inds} \hlkwb{<-} \hlkwd{order}\hlstd{(K[i, ])[}\hlnum{1}\hlopt{:}\hlnum{8} \hlopt{+} \hlnum{1}\hlstd{]}
  \hlstd{H[i,inds]} \hlkwb{<-} \hlstd{H[inds, i]} \hlkwb{<-} \hlnum{1}
\hlstd{\}}
\end{alltt}
\end{kframe}
\end{knitrout}

To check any player's ``neighbors'' according to \texttt{H}, we can do (for Al Horford):

\begin{knitrout}
\definecolor{shadecolor}{rgb}{0.969, 0.969, 0.969}\color{fgcolor}\begin{kframe}
\begin{alltt}
\hlstd{this.player} \hlkwb{<-} \hlkwd{grep}\hlstd{(}\hlstr{"Horford"}\hlstd{, players}\hlopt{$}\hlstd{lastname)}
\hlkwd{paste}\hlstd{(players}\hlopt{$}\hlstd{firstname, players}\hlopt{$}\hlstd{lastname)[}\hlkwd{which}\hlstd{(H[this.player, ]} \hlopt{==} \hlnum{1}\hlstd{)]}
\end{alltt}
\begin{verbatim}
##  [1] "Brandon Bass"      "J.J. Hickson"      "Andre Drummond"   
##  [4] "Tony Mitchell"     "David Lee"         "Dwight Howard"    
##  [7] "Blake Griffin"     "Zach Randolph"     "Anthony Davis"    
## [10] "Amar'e Stoudemire" "Jason Maxiell"     "Glen Davis"       
## [13] "DeMarcus Cousins"  "Jonas Valanciunas" "Enes Kanter"
\end{verbatim}
\end{kframe}
\end{knitrout}

Similarly, let's load the basis functions that are used in representing the spatial effects in players' macrotransition entry models: we denote these basis functions $\phi_{ji}$, where $i=1, \ldots, 10$, and $j$ indexes shot-taking, four different pass options, and turnovers (recall that for the spatial effects in the shot probability model (Equation 10 in the paper), we use the same basis functions as we do for the shot-taking hazard model). To recreate Figure 6 of the paper, which plots the shot-taking bases, we'd do:

\begin{knitrout}
\definecolor{shadecolor}{rgb}{0.969, 0.969, 0.969}\color{fgcolor}\begin{kframe}
\begin{alltt}
\hlkwd{source}\hlstd{(}\hlkwd{sprintf}\hlstd{(}\hlstr{"%s/model_util.R"}\hlstd{, code.dir))}
\hlkwd{par}\hlstd{(}\hlkwc{mfrow}\hlstd{=}\hlkwd{c}\hlstd{(}\hlnum{2}\hlstd{,}\hlnum{5}\hlstd{))}
\hlkwa{for}\hlstd{(i} \hlkwa{in} \hlnum{1}\hlopt{:}\hlnum{10}\hlstd{)}
  \hlkwd{spatialPlot1}\hlstd{(take.basis[i, ],} \hlkwc{legend}\hlstd{=F)}
\end{alltt}
\end{kframe}\begin{figure}[h!]

{\centering \includegraphics[width=\maxwidth]{figure/shot_bases-1} 

}

\caption[Shot-taking spatial bases]{Shot-taking spatial bases; this plot is the same as Figure 6 of the paper (though the ordering is different).}\label{fig:shot_bases}
\end{figure}


\end{knitrout}

\section{Loading parameters and model estimates}

Here, we will load and illustrate the results of the multiresolution transition models discussed in Section 3 of the paper. First, let's load the (offensive) microtransition model output for LeBron James, print the parameter estimates, and plot of the acceleration effects $\mu^{\ell}_x, \mu^{\ell}_y$, as in Figure 4 of the paper.

\begin{knitrout}
\definecolor{shadecolor}{rgb}{0.969, 0.969, 0.969}\color{fgcolor}\begin{kframe}
\begin{alltt}
\hlstd{player.id} \hlkwb{<-} \hlstd{players}\hlopt{$}\hlstd{player_id[}\hlkwd{which}\hlstd{(players}\hlopt{$}\hlstd{firstname} \hlopt{==} \hlstr{"LeBron"} \hlopt{&} \hlstd{players}\hlopt{$}\hlstd{lastname} \hlopt{==} \hlstr{"James"}\hlstd{)]}
\hlkwd{load}\hlstd{(}\hlkwd{sprintf}\hlstd{(}\hlstr{"%s/micros/%s.Rdata"}\hlstd{, data.dir, player.id))}
\hlstd{with.ball}\hlopt{$}\hlstd{io.x}\hlopt{$}\hlstd{summary.fixed[,} \hlnum{1}\hlopt{:}\hlnum{5}\hlstd{]} \hlcom{# x component of LeBron James' micro model during ball possession}
\end{alltt}
\begin{verbatim}
##                   mean           sd  0.025quant     0.5quant 0.975quant
## dif       0.9794630172 0.0005972388  0.97829043 0.9794630009 0.98063462
## intercept 0.0001920039 0.0123200077 -0.02730304 0.0001092341 0.02799413
\end{verbatim}
\begin{alltt}
\hlkwd{par}\hlstd{(}\hlkwc{mfrow}\hlstd{=}\hlkwd{c}\hlstd{(}\hlnum{1}\hlstd{,}\hlnum{2}\hlstd{))}
\hlkwd{par}\hlstd{(}\hlkwc{mar}\hlstd{=}\hlkwd{c}\hlstd{(}\hlnum{0}\hlstd{,}\hlnum{0}\hlstd{,}\hlnum{0}\hlstd{,}\hlnum{0}\hlstd{))}
\hlkwd{vectorPlot}\hlstd{(with.ball)}
\hlkwd{vectorPlot}\hlstd{(without.ball)}
\end{alltt}
\end{kframe}\begin{figure}[h!]

{\centering \includegraphics[width=.5\linewidth]{figure/micro-1} 

}

\caption[Plots of acceleration effect for LeBron James' offensive microtransition model]{Plots of acceleration effect for LeBron James' offensive microtransition model.}\label{fig:micro}
\end{figure}


\end{knitrout}

The defensive microtransition model is less complicated, and we can fit it very quickly. The code below estimates the same model parameters for all players on defense:

\begin{knitrout}
\definecolor{shadecolor}{rgb}{0.969, 0.969, 0.969}\color{fgcolor}\begin{kframe}
\begin{alltt}
\hlstd{def.micro} \hlkwb{<-} \hlkwd{microDefModel}\hlstd{(new.dat)}
\hlcom{# coefficients are a_x, c_x, and b_x from Equation 6 in paper}
\hlkwd{summary}\hlstd{(def.micro}\hlopt{$}\hlstd{mod.x)}\hlopt{$}\hlstd{coef[,} \hlnum{1}\hlopt{:}\hlnum{3}\hlstd{]}
\end{alltt}
\begin{verbatim}
##                                      Estimate   Std. Error    t value
## (Intercept)                     -0.0013986229 6.612938e-05  -21.14979
## def.eps.x[-length(def.eps.x)]    0.9816341676 3.841023e-04 2555.65802
## residual.x[-length(residual.x)] -0.0001942118 6.656132e-06  -29.17788
## opt.eps.x[-length(opt.eps.x)]    0.0089958396 3.508471e-04   25.64034
\end{verbatim}
\end{kframe}
\end{knitrout}



% show a micro, a macro entry INLA thing, and a tmat

% load inla outputs and transition matrices




\end{document}
